\documentclass{report}

\usepackage{graphicx}
\usepackage{color}
\usepackage[utf8]{inputenc}
\usepackage[T1]{fontenc} 
\usepackage[francais]{babel} 
\usepackage{amssymb}
\usepackage{amsmath}
\usepackage{amstext}
\usepackage{amsfonts}
\usepackage{amsthm}
\usepackage[francais]{layout}

\newcommand{\R}{\mathbb{R}}
\newcommand{\C}{\mathbb{C}}
\newcommand{\N}{\mathbb{N}}
\newcommand{\Z}{\mathbb{Z}}
\newcommand{\Q}{\mathbb{Q}}
\newcommand{\K}{\mathbb{K}}



\newcommand{\deffinition}{\textcolor[rgb]{0.65,0,0.7}{Définition : }}
\newcommand{\Def}{\textbf{\deffinition}}

\newcommand{\preuve}{\textcolor[rgb]{0.95,0.25,0}{Preuve : }}
\newcommand{\Preuve}{\textbf{\preuve}}

\newcommand{\propo}{\textcolor[rgb]{0,0,0.75}{Proposition : }}
\newcommand{\Propo}{\textbf{\propo}}

\newcommand{\remarque}{\textcolor[rgb]{0,0.65,0.05}{Remarque : }}
\newcommand{\RQ}{\textbf{\remarque}}

\newcommand{\thm}{\textcolor[rgb]{0,0.4,1}{Théorème : }}
\newcommand{\THM}{\textbf{\thm}}

\newcommand{\lemme}{\textcolor[rgb]{0,0.2,0.80}{Lemme : }}
\newcommand{\Lemme}{\textbf{\lemme}}

\newcommand{\coro}{\textcolor[rgb]{0.3,0,0.70}{Corolaire : }}
\newcommand{\Coro}{\textbf{\coro}}

\newcommand{\vocable}{\textcolor[rgb]{0,0,0.50}{Vocable : }}
\newcommand{\Vocable}{\textbf{\vocable}}




\title{Théorie des Graphes}
\author{Lucas Moragues}
%\date{}
\begin{document}
\maketitle
\renewcommand{\contentsname}{Sommaire}
\setcounter{tocdepth}{3}
\tableofcontents
\chapter{Graphe et combinatoire}
\setcounter{section}{-1}
\section{Notion sur les graphes}											%\addcontentsline{toc}{section}{Some Obvious things}
\Def\\
Un graphe simple est un couple $G=(V,E)$ avec $V$ l'ensemble des sommets $E$ l'ensemble des arêtes qui sont des paires de sommets distincts\\
$$E \subseteq \{\{u,v\}|u,v \in V ; u \neq v\}$$

\Vocable\\
$u$ et $v$ sont les extrémités de l'arête $\{u,v\}$\\
L'arête $e =\{u,v\}$ est uncidente à $u$ et $v$\\
Le degré d'un sommet $v$ est le nobmre d'arêtes incidente à $v$\\\\

\textbf{\lemme lemme des poignées de main}\\
Soit $G=(V,E)$ un graphe simple alors la somme des degrès des sommets de $G$ est deux fois le nombre d'arêtes\\ 
$$\sum_{v\in V}degre(v)=2|E|$$
Le degré minimum est noté $\delta(G)$\\
$$\delta(G)=Min_{v\in V} degre(v)$$
Le gegré maximum est noté $\Delta(G)$\\
$$\Delta(G)=Max_{v\in V} degre(v)$$

\section{Cycle chemin et connexité}

Soit $u,v \in V$ un chemin de $u$ à $v$ dans $G=(V,E)$ est une séquence d'arêtes telle que deux arêtes consécutives aient une extrémité commune et tel que la première arête sont incidente à u et la dernière incidente a $v$ $(\{U_0,U_1\},\{U_1,U_2\},...,\{U_{k-1},U_k\})$\\

La distance entre deux sommets $u,v \in V$ est la longueur du plus cours chemin $u$ et $v$ on notera cette valeur $dist(u,v)$\\
Le diamète d'un graphe, noté $diam(G)$ est la distance entre ses deux sommets les plus éloignés\\
$$diam(G)=max_{u,v \in V} dist(u,v)$$
Le rayon d'un graphe est défini par :\\
$$Rayon(G)= min_{u\in V} max_{v \in V} dist(u,v)$$
Un angle est un chemin dans les deux extrémités sont identiques\\
Un graphe est connexe si il existe un chemin entre toute paire de sommet\\\\

\Def\\
Soit $G=(V,E)$ un graphe et $U \subseteq V$ un sous-ensemble de sommets, le sous-graphe induit par $U$ noté $G_{|U}$ ($G$ restreint à $U$) est le graphe dont les sommets sont $U$ et dont les arêtes sont celles de $E$ dont les extrémités sont dans $U$\\
$$G_{|U}=(u,\{\{u,v\}|\{u,v\}\in E; u,v \in U\}$$
Un sous-graphe induit connexe maximal (pour sont support ) est une \textbf{composante} (connexe)\\
Une \textbf{clique} est un ensemble de sommets qui sont tous adjacents\\
Deux sommets sont adjacents s'ils sont incidents à une même arête. $u$ et $v$ adjacent si $\{u,v\} \in E$\\

\begin{figure}[h]
	\centering
		\includegraphics{descliques}
	\caption{On note $K_n$ la clique avec $n$ sommets}
	\label{descliques}
\end{figure}

Un graphe $G=(V,E)$ est biparti si on peut partitionner $V$ en deux ensemble $A$ et $B$ tel que toute arête de $E$ aient une extrimité dans $A$ et dans $B$\\

\Lemme\\
si $G=(V,E)$ est connexe, $|E|\geq|V|-1$\\

\Preuve\\
Soit $v\in V, \forall w \in V\backslash_{\{v\}}$ on note $P_w$ un plus cours chemin de $w$ a $v$ et $e_w$ la première arête de  $P_w$\\\\
si $w\neq w'$ alors $e_w\neq e_{w'}$\\
On suppose que cette affirmation est fausse\\
Soit $w\neq w'$ telque $e_w=e_{w'}$\\
$$dist(w,v)=1+l'\leq l$$
$$dist(w',v)1+l\leq l'$$

\Lemme\\
Soit $G=(V,E)$  un graphe connexe et $e \in E/ G\backslash_{\{e\}}$ est connexe si et seulement si e appartien à un cycle dans $G$\\

\Preuve\\
$\Rightarrow$\\
$G\backslash_{\{e\}}$ connexe. Soit $e=\{u,v\}$\\
il exite un chemin $P$ de $u$ à $v$ dans $G-e$ Donc l'union de $P$ et de $e$ forme un cycle dans $G$ qui passe par $e$\\
$\Leftarrow$\\
Soit $C$ un cycle qui contient $e=uv$\\
il existe un chemin dans $G-e$ entre $u$ et $v$\\
soit $u',v' \in V$ soit $P$ un chemin de $u'$ à $v'$ dans $G$ si $e\notin P$ alors $u'$ et $v'$ sont connecté dans $G-e$\\

Si $e \in P$ sans perte de généralité on suppoe que $u'$ est connecté a $u$ et $v'$ et connecté a $v$ dans la relation de connexion est transitive donc $u'$ est connecté à $v'$ dans $G-e$\\
Donc $G-e$ est connexe\\

\begin{figure}[h]
	\centering
		\includegraphics{chemins}
	\caption{On note $P_n$ le chemin de $n$ sommets}
	\label{chemins}
\end{figure}

\begin{figure}[h]
	\centering
		\includegraphics{cycles}
	\caption{On note $C_n$ le cycle de $n$ sommets}
	\label{cycles}
\end{figure}

\begin{figure}[h]
	\centering
		\includegraphics{graphe_biparti}
	\caption{On note $K_{n,m}$ le graphe biparti complet avec $n$ sommets d'un coté et $m$ de l'autre}
	\label{graphe_biparti}
\end{figure}

\subsection{Complement sur les chemins}

on défini la relation $\sim$ telle que $u \sim v$ si $u$ et $v$ sont connecté par un chemin ,( même un chemin vide $\Rightarrow u \sim v$ )\\

\Propo\\
$\sim $ est une relation transitive\\

\Preuve\\
soit $u \sim v$ et $v \sim w$ soit $P$ un chemin de $u$ à $v$, soit $Q$ un chemin de $v$ à $w$ \\
Soit $s$ le premier sommet visité par $P$ de $u$ à $v$ et $Q$ de $w$ à $v$\\
on peut donc construir le chemin $R$ avec les arêtes de $P$ qui vont de $u$ à $s$ plus les arêtes de $Q$ qui vont de $s$ à $w$\\
\newpage
\Propo\\
$\sim$ est une relation d'équivalence
reflexive : $u \sim u$\\
transitive : $u \sim v \wedge v \sim w\Rightarrow u \sim w$\\
symérique : $u \sim v\Rightarrow v \sim u$\\

Les classes d'équivalence de $\sim$ sont les composantes (connexe) du graphe\\
On note c(G) le nombre d conposantes connexes d'un graphe G\\

\section{Graphes minimalement connexes}
\Def\\
\begin{itemize}
	\item un graphe $G=(V,E)$ est minimalement connexe si , $G$ est connexe et pour tout $e \in E,G-e$ n'est plus connexe
	\item un arbre est un graphe minimalement connexe
	\item une forêt est un graphe sans cycle\\
\end{itemize}

\Propo\\
Pour toute forêt $G=(V,E)$\\
$$\left|V\right|=\left|E\right|+c(G)$$

\Preuve\\
On va montrer cette propriété par indection sur le nombre d'arêtes\\
Soit $G$ un graphe avec $n$ sommets et $m=0$ arêtes alors $c(G) =n \Rightarrow \left|V\right|=\left|E\right|+c(G)$\
Soit $G$ un graphe avec $n$ sommets et $m\geq1$arêtes acyclique\\
Soit $u,v \in E$ $u$ et $v$ sont connectés dans $G$ et déconnecté dans $G-e$ car G est acyclique \\
$c(G) = c(G-e)-1$\\

par induction :\\
$\left|V\right|=\left|E-e\right|+c(G-e)$\\
$\left|V\right|=\left|E\right|-1+c(G-e)$\\
$\left|V\right|=\left|E\right|+c(G)$\\

\Propo\\
Soit $G=(V,E)$ un graphe connexe il existe $E'\subseteq E$  tel que $G'=(V,E')$ est connexe acyclique\\
Si il y à un cycle dans $G$ et $e$ appartient à ce cycle alors $G-e$ est connexe\\
On peu appliquer cette propriété tant qu'il existe un cycle\\
\newpage
\THM\\
Soit $G=(V,E)$, les proposition suivante sont équivalente\\
\begin{enumerate}
	\item $G$ est un arbre (minimalement connexe)
	\item $G$ est connexe et sans cycle
	\item $G$ est acyclique et $\left|E\right|=\left|V\right|-1$
	\item $G$ est connexe et $\left|E\right|=\left|V\right|-1$
	\item $G$ est maximal sans cycle
	\item pour tout $u\neq v \in V,G$ contient un unique chemin de $u$ à $v$\\
\end{enumerate}

\Preuve\\
\textbf{$1)\Rightarrow5) :$} Si on  ajoute $u,v \notin E à G$ , il existe $P$ un chemin de $u$ à $v$ $P+uv$ est donc un cycle $G$ et est maximal sans cycle\\
\\\textbf{$5)\Rightarrow2) :$} Par contradiction, si $G$ n'est pas connexe , alors il existe $u$ déconnecté de $v$ $G+uv$ est acyclique donc $G$ n'est pas maximal sans cycle\\
\\\textbf{$2)\Rightarrow3) :$} $\left|V\right|=\left|E\right|+c(G)$ $G$ est connexe $c(G)=1 \Longrightarrow \left|E\right|=\left|V\right|-1$\\
\\\textbf{$3)\Rightarrow4) :$} $G$ acyclique $\Rightarrow \left|V\right|=\left|E\right|+c(G)$ et $\left|E\right|=\left|V\right|-1 \Rightarrow c(G)=1$\\
\\\textbf{$4)\Rightarrow6) :$} Soit $G$ connexe et $\left|E\right|=\left|V\right|-1$ On suppose que $u$ et $v$ sont relier par deux chemin . $G'=(V,E')$ un sous graphe acyclique connexe $\left|E'\right|=\left|V\right|-1$ et $\left|E'\right|<\left|E\right|$ D'où contradiction Il reste un unique chemin de $u$ à $v$\\
\\\textbf{$6)\Rightarrow5) :$}	$G$ est connexe et $G-uv$ est déconnecté car l'arête $uv$ est l'unique chemin de $u$ à $v$\\

\Propo\\

missed something here\\

\Def\\
Un graphe k régulier est un graphe dont tous les sommets sont de degrès k\\
$$\sigma(G)=\Delta(G)=k$$
\newpage
Caractérisation des graphes bipartis

\Propo\\
Un graphe G=(V,E) est biparti si et seulement si il n'a pas de cycle de longueur impaire\\
G sans cycle impair connexe $\Rightarrow$ G biparti
\section{Euler et Hamilton}
Une chaîne est un chemin qui peut passer plusieurs fais par un même sommet mais une seule fois par arête\\
Une chaîne fermer au circuit est une chaîne dont les deux extrémités sont identique\\
Une chaîne Eulerienne est une chaîne qui passe exactement une fois par chaque arête . Un circuit Eulérien est un circuit qui passe exactement une fois par chaque arête. un graphe est Eulérien si il a un circuit Eulérien\\

\THM\\
Soit G=(V,E) connexe avec au moin 2 sommets , les 3 propriétés sont équivalentes\\
\begin{enumerate}
	\item G est Eulérien
	\item degrès($v$) est pair pour tout $v \in V$
	\item il y a un ensemble de cycles C tel que toute arête soit exactement dans un de ces cycles\\
\end{enumerate}
\Preuve\\
\textbf{$i)\Rightarrow ii) :$} soit $W$ un circuit Eulérien soit $v \in V$ chaque occurence de $v$ dans $W$ utilise deux arêtes incitente à $v$ . $v$ est donc de degrès pair\\
\\\textbf{$ii)\Rightarrow iii) :$} soit $G$ tel que tous ses sommets soit de degrès pair On choisit un sommet $v$ et on construit un chemin maximal à partir de $v,P$. le dernier sommet de $P$ est de degrès $\geq 2$ et donc il est adjacent à un sommet $w$ de $P$. La partie de $P$ de $w$ à $v$ avec l'arête $wv$ forme un cycle $C$ Soit $G-C$ le graphe $G$ auquel on a enlever les arêtes de $C$ le degrès des somets de $G-C$ reste pair par induction on peut décomposer $G-C$ en cycle\\
\\\textbf{$iii)\Rightarrow i) :$} soit $T$ un circuit maximal contituer d'un ensemble de cycle $C_1,...,C_n$\\
si il existe un autre cycle de $C$ alors comme $G$ est connexe il existe un cycle de $C$ qui n'est pas dans $T_v$ mais qui partage un sommet $v$ avec $T$. On note $C$ ce cycle\\ $C(vu_1,...,u_kv)$, $T(vw_1,...,w_kv)$ donc  $(vu_1,...,u_kv,vw_1,...,w_kv)$ est un circuit)\\
ce qui contredit la maximalité de $T$, $T$ contient donc tous les cycles de $C$ et ainsi toutes les arêtes de $G$ . $G$ est Eulérien\\

\end{document}